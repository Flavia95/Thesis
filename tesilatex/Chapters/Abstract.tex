% Chapter discussion

\chapter{Abstract-1pagina}
    
 % Main chapter title

\label{Chapter0} % For referencing the chapter elsewhere, use \ref{Chapter4} 

%----------------------------------------------------------------------------------------

% Define some commands to keep the formatting separated from the content 
\newcommand{\keyword}[1]{\textbf{#1}}
\newcommand{\tabhead}[1]{\textbf{#1}}
\newcommand{\code}[1]{\texttt{#1}}
\newcommand{\file}[1]{\texttt{\bfseries#1}}
\newcommand{\option}[1]{\texttt{\itshape#1}}

%~~~~~~~~~~~~~~~~~~~~~~~~~~~~~~~~~~~~~~~~~~~~~~~~~~~~~~~~~~~~~~~~~~~~~~~~~~~~~~~~~~~~~~~
Studies of genomic selection typically assume a single linear reference genome. However, structural and complex variation can render this simplified model inapplicable. To address this limitation, during my thesis I started the development of a software library for the statistical analysis of negative selection using pangenomic data models. Typically represented in the Graphical Fragment Assembly (GFA) file format, these models represent whole genome alignments in a compact structure, without loosing information. Because a pangenome embeds the linear genomes from which it is constructed, we can choose a particular reference genome and project the variants on it. """ DIRE CHE E' RAPPRESENTATO COME UN GRAFO E POI CHE LE ZONE CONTENENTI VARIANTI APPAIONO COME BOLLE(which appear as bubbles in the graph)""". Therefore, I focused on develop algorithms for bubble detection, genereting Variant Call Format (VCF) files directly from graphs.\\

\noindent
We can use this variants projection to drive standard population genetic analyses. In my thesis, I focused on the calculation of Fst for three different time.

